%%%%%%%%%%%%%%%%%%%%%%%%%%%%%%%%%%%%%%%%%%%%%%%%%%
%%%%% DNA Seq Package %%%%%%%%%%%%%%%%%%%%%%%%%%%% % https://tex.stackexchange.com/questions/723487/modification-for-use-of-dnaseq-package
%%%%%%%%%%%%%%%%%%%%%%%%%%%%%%%%%%%%%%%%%%%%%%%%%%

\usepackage{dnaseq}                       

\renewcommand{\ttdefault}{lmtt}                   % for distinctive boldface

\makeatletter

%%% remove the awkward definition of `\struty`
\def\DNA#1{%
   \def\@DNA@end{#1}\bgroup\ttfamily\DNAc@lcline
   %% dnabase per line counter
   \count@=0
   %% block counter
   \@tempcnta=0
   %% total dnabase counter
   \@tempcntb=0
   \fboxrule=0pt \fboxsep=0pt
   \noindent\phantom{\DNAreserve}\llap{1}\ %
   \@DNA
}
%%% a better strut
\def\DNA@strut{%
  \vrule height \dimeval{\fontcharht\font`I+2pt} depth 1pt width 0pt
}

%%% fix Steven's code
\def\@DNA#1{%
    %% insert a space after \DNAblock bases
    \ifnum\count@=\DNAblock\count@=0\ %
    \advance\@tempcnta by 1\fi
    \def\@DNA@cmp{#1}%
    %% check for end of sequence or color shift
    \ifx\@DNA@cmp\@DNA@end
    \let\next\egroup
    \else
    \ifx\@DNA@cmp\@DNA@color
    \let\next\@DNA@setcolor
    \else
    \advance\count@ by 1
    \advance\@tempcntb by 1
    %% line break after calculated number of blocks
    \ifnum\@tempcnta=\blocks \\
    \hskip\z@\phantom{\DNAreserve}\llap{\the\@tempcntb}\ %
    \@tempcnta=0
    \fi
    \makebox[0.5em]{\@DNA@thecolor{\DNA@strut#1}}% <--- here the changes
    \penalty0\let\next\@DNA
    \fi
    \fi
    \next
}
\makeatother

% how to use: 
% 
% \DNA!
%     '{} ATGCATCG...
% !

%%%%%%%%%%%%%%%%%%%%%%%%%%%%%%%%%%%%%%%%%%%%%%%%%%
%%%%% Insert R code in LaTeX %%%%%%%%%%%%%%%%%%%%% % https://tex.stackexchange.com/questions/374001/insert-r-code-in-latex
%%%%%%%%%%%%%%%%%%%%%%%%%%%%%%%%%%%%%%%%%%%%%%%%%%

\usepackage{listings}
\usepackage[dvipsnames]{xcolor}

\definecolor{codegreen}{rgb}{0,0.6,0}
\definecolor{codegray}{rgb}{0.5,0.5,0.5}
\definecolor{codepurple}{rgb}{0.58,0,0.82}
\definecolor{backcolour}{rgb}{0.95,0.95,0.92}

\lstdefinestyle{mystyle}{
    backgroundcolor=\color{backcolour},   
    commentstyle=\color{codegreen},
    keywordstyle=\color{magenta},
    numberstyle=\tiny\color{codegray},
    stringstyle=\color{codepurple},
    basicstyle=\ttfamily\footnotesize,
    breakatwhitespace=false,         
    breaklines=true,                 
    captionpos=b,                    
    keepspaces=true,                 
    numbers=left,                    
    numbersep=5pt,                  
    showspaces=false,                
    showstringspaces=false,
    showtabs=false,                  
    tabsize=2
}
\lstset{style=mystyle}
